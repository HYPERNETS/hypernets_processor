\subsection{Scope and objectives}
The land HYPERNETS processor (hereafter termed HYPROCESSOR) describes the software component of the HYPERNETS network.
The HYPROCESSOR is designed to convert the raw data collected from the measurement network under the standard measurement protocol to the designated products.
The products being produced, as well as the manner of their production, are described in this document.
This document details the theoretical and practical implementation of the methods used.

\subsection{Algorithm justification}
The HYPROCESSOR software aims to convert the raw digital numbers (DN) produced by the network instruments to hemispherical-conical reflectance factors (HCRF) with associated uncertainty information based on the calibration and characterisation of the individual instruments and their operating conditions.
This refers solely to the standard mode of operation of the network which is defined by the standard measurement protocol (a sequence of measurements that every Hypernets sensor operating within the network must make).
Additional protocols may be implemented by site users but processing of this data does not fall under the standard processing.
However, there are various reflectance quantities of interest, given the array of surface reflectance products that exist.

\subsubsection{Quantities of interest}
A thorough review of reflectance quantities can be found in \textbf{Nicodemus}, and a version translated into a remote sensing context in \textbf{Schaepman-Strub et al}.

\subsubsection*{Hemispherical-conical reflectance factor (HCRF)}
The main quantity that will be measured directly by the Hypernets instrument is HCRF.
This quantity includes integration of the full illumination hemisphere and a conical subsection of the reflectance hemisphere defined by the nominal field of view of the instrument and the zenith and azimuth angles of its centre.
The formal definition provided in \textbf{Schaepman-Strub et al} is given as:

\begin{equation}
Eq here
\label{HCRF_def}
\end{equation}

\subsubsection*{Bidirectional reflectance distribution function}
The Bidirectional Reflectance Distribution Function (BRDF) is an intrinsic property of a surface and the theoretical quantity from which all other reflectance quantities can be derived \textbf{Schaepman-Strub}.
The BRDF is a function that describes the directional reflectance of a surface from any combination of incident and exitant zenith and azimuth angles.
Since BRDF is predicated on directional reflectance, it cannot be measured due to the infinitesimally small angles over which it is defined, only approximated by sampling multiple illumination and viewing geometries.
The BRDF is formally defined by \textbf{Schaepman-Strub} as:
\begin{equation}
Eq. here
\label{BRDF_def}
\end{equation}

%\subsubsection*{Bi-hermispherical reflectance}

\subsubsection{Uncertainty}
A key feature of the HYPROCESSOR is the need to provide uncertainty information associated with the measurement quantities produced, as well as the measurements being traceable to SI.
The instrument and characterisation tests related to this are discussed in separate documents.
However, the propagation of this information through the processing chain is given alongside the product formulation.